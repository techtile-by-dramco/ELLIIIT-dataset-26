% usage:
% 1. define \author{gilles, author2, author3} after begin document
% 2. optinally you can print the commands by provding a true, false option:
%        \author[false]{gilles, author2, author3}, this will not display an overview of the avaiable commands, by default it does.
% 3. use the created commands, e.g., \gilles{this a colored remark}, this outputs "[gilles: this a colored remark]" in an author-unique color.

 
 
 
 % Required to specify custom colors
\makeatletter
% \@ifpackageloaded{xcolor}{%
\PassOptionsToPackage{dvipsnames,svgnames,table}{xcolor}%
% }{%
% \usepackage[dvipsnames,svgnames, table]{xcolor}%
% }
\makeatother


\usepackage{listofitems}
\usepackage{pgffor}
\usepackage{xparse}
\usepackage{ifthen}
\usepackage{tikz}
\usepackage{xspace}
\usetikzlibrary{external}

\makeatletter
% Define your own list of colors
\def\myColorList{LimeGreen,BrickRed,Fuchsia,Bittersweet,YellowOrange, YellowGreen, WildStrawberry, Fuchsia, RedViolet, Periwinkle, PineGreen, OrangeRed, RawSienna, Turquoise, Aquamarine, BlueViolet, BurntOrange, DarkOrchid, TealBlue, Violet, RoyalPurple,%
% new colors to be found
LimeGreen,BrickRed,Fuchsia,Bittersweet,YellowOrange, YellowGreen, WildStrawberry, Fuchsia, RedViolet, Periwinkle, PineGreen, OrangeRed, RawSienna, Turquoise, Aquamarine, BlueViolet, BurntOrange, DarkOrchid, TealBlue, Violet, RoyalPurple}
\readlist\ColorList\myColorList


\definecolor{defaultColor}{RGB}{112,45,128}
\NewDocumentCommand{\roundLabel}{O{defaultColor} m}{%
\tikzexternaldisable
    \tikz[baseline=(char.base)]{
        \node[rectangle, rounded corners=0.65mm, inner sep=0.65mm, fill=#1, draw=#1, text=white, font=\itshape](char) {#2};
    }%
\tikzexternalenable
}


% Define the command with an optional parameter
\NewDocumentCommand{\defineauthors}{ O{true} m }{%
    \ifthenelse{\equal{#1}{true}}{%
        % Code for true
        Personal commands for comments are:
        \begin{itemize}
        \foreach \x [count=\xi from 1] in {#2} {   
            \item \textcolor{\ColorList[\xi]}{\textbackslash\unexpanded\expandafter{\x}\{\}}
            }
        \end{itemize}
        \foreach \x [count=\xi from 1] in {#2} { 
            \expandafter\xdef\csname\x\endcsname####1{\noexpand\textcolor{\ColorList[\xi]}{\roundLabel[\ColorList[\xi]]{\unexpanded\expandafter{\x}} ####1}}%
        }
    }{%
        % Code for false
        \foreach \x [count=\xi from 1] in {#2} { 
            \expandafter\xdef\csname\x\endcsname####1{\noexpand\textcolor{\ColorList[\xi]}{\roundLabel[\ColorList[\xi]]{\unexpanded\expandafter{\x}} ####1}}%
        }
    }
    \foreach \x [count=\xi from 1] in {#2} {
        \expandafter\xdef\csname at\x\endcsname####1{\noexpand\textcolor{\ColorList[\xi]}{@\x}}%
    }
}

\NewDocumentCommand{\removeauthors}{ m }{%
    \foreach \x in {#1} { 
        \expandafter\edef\csname\x\endcsname####1{}%
    }
    \foreach \x in {#1} {
        \expandafter\edef\csname at\x\endcsname####1{}%
    }
}



% \newcommand{\defineauthors}[1]{
%     \foreach \x [count=\xi from 1] in {#1} { 
%     \expandafter\xdef\csname\x\endcsname####1{\noexpand\textcolor{\ColorList[\xi]}{[\unexpanded\expandafter{\x}: ####1]}}%
%     }
% }
\makeatother

